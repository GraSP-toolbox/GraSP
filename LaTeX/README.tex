To use these packages, either copy them in your Latex source directory, or add this directory to 
your TEXINPUTS environment variable.

You will need the csvtools package (http://www.ctan.org/pkg/csvtools). A migration to datatool is 
on the roadmap for a future release of GraSP.

These packages are designed to work with the output of grasp_exportcsv and grasp_exportcsv_signal, 
but any similar csv files will work.

The packages here are compatible with babel and tikz/externalize (highly recommended).

The packages tikzbabel and tikzcolorscale are automatically loaded by tikzgraph and tikzmatrix and 
are merely helper packages.

== tikzgraph ==

To use this package simply add
  \usepackage{tikzgraph}
in the preamble.

To plot a simple graph signal use the following code (default options are given):

\begin{tikzpicture}
  \flatgraph[%
      maxcol=black,        % Color of the maximum value
      mincol=white,        % Color of the minimum value
      edgecol=gray,        % Color of the edges
      nodeedgecol=black,   % Color of the edges of a vertex
      maxvalue=1.0,        % Maximum value of samples of the signal
      minvalue=0.0,        % Minimum value of samples of the signal
      nodesize=1opt,       % Size of a vertex
      nodelinewidth=0.2pt, % Thickness of the edges of a vertex
      intprecision=2,      % Number digits after the floating point
      textscale=1,         % Scale of the text in the figure (mostly the color scale boundaries)
      xscale=1,            % Width / height ratio
      colormap=hot,        % One of {hot, jet, bicolor, redblue, redbluedark, hsv}
      showcolormap=false,  % Whether the color map LUT is shown
      showedges=true]      % Whether edges are shown
    {nodes.csv}%
    {edges.csv}%
    {signal.csv}
\end{tikzpicture}

To add a background image use:

\begin{tikzpicture}
  \backgroundflatgraph[%
      <options of flatgraph>,%
      backgroundwidth=10cm, % Width of the background
      appendedcode={}]      % Code to append after the figure definition (such that it gets 
scaled)
    {nodes.csv}%
    {edges.csv}%
    {signal.csv}%
    {backgroundimage}
\end{tikzpicture}

To put the graph on a slanted plane and plot the signal using stems, use:

\begin{tikzpicture}
  \stemgraph[%
      <options of flatgraph>,%
      zangle=0,    % How slanted is the plane supporting the graph
      stemscale=1, % Scaling of stems (i.e. values of vertices)
      poscol=red,  % Color of positive values
      negcol=blue] % Color of negative values
    {nodes.csv}%
    {edges.csv}%
    {signal.csv}
\end{tikzpicture}

== tikzmatrix ==

To use this package simply add
  \usepackage{tikzmatrix}
in the preamble.

To show a matrix with N columns, use:

\begin{tikzpicture}
  \tikzmatrix[%
      maxcol=black,         % Color of the maximum value
      mincol=white,         % Color of the minimum value
      maxvalue=1.0,         % Maximum value of samples of the signal
      minvalue=0.0,         % Minimum value of samples of the signal
      intprecision=2,       % Number digits after the floating point
      textscale=1,          % Scale of the text in the figure (mostly the color scale boundaries)
      colormap=hot,         % One of {hot, jet, bicolor, redblue, redbluedark, hsv}
      showcolormap=false,   % Whether the color map LUT is shown
      showinnerlines=false] % Whether to show edges separating elements of the matrix
    {matrix.csv}%
    {N}
\end{tikzpicture}

